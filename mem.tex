\documentclass[a4paper, 11pt]{article}
\usepackage[utf8]{inputenc}
\usepackage[spanish]{babel}
\usepackage{amsmath}
\usepackage{amsfonts}
\usepackage{amssymb}
\usepackage{listings}
\usepackage{color}
\usepackage{hyperref}
\usepackage{graphicx}
\usepackage{enumitem}
\usepackage{geometry}
\geometry{
  a4paper,
  left=20mm,
  right=20mm,
  top=20mm,
  bottom=20mm,
}
\lstset{
    basicstyle=\footnotesize\ttfamily,
    breaklines=true,
    columns=fullflexible,
}
\setlist{noitemsep}
\setlength{\parskip}{0.5em}
\setlength{\parindent}{0em}

\title{Memoria de la Practica de Parchis}
\author{Jesús Losada Arauzo}
\date{\today}

\begin{document}

\maketitle

\section{Analisis del problema}
En esta práctica se debe hacer una variante del parchis. Se puede elegir los dados y el 3 no existe. Los dados se restauran una vez que se han gastado todos. Además, hay un dado especial con varios movimientos especiales.

El objetivo es ganarle a los 4 niveles (ninjas), implementando una Poda Alfa-Beta o Minimax, y una heurística eficaz.

\section{Descripción del problema}

\subsection{Poda Alfa-Beta}
Aquí se muestra la Poda Alfa-Beta implementada:

\begin{lstlisting}[language=C++]
double AIPlayer::Poda_AlfaBeta(
    const Parchis &actual, int jugador, 
    int profundidad, int profundidad_max, 
    color &c_piece, int &id_piece, 
    int &dice, double alpha, 
    double beta, 
    double (*heuristic)(const Parchis &, int)
) const {
    if (profundidad == profundidad_max || actual.gameOver())
        return heuristic(actual,jugador);

    bool Max_verstappen = actual.getCurrentPlayerId() == jugador;
    double valor;
    ParchisBros hijos = actual.getChildren(); 

    for (auto it = hijos.begin(); it != hijos.end(); ++it) {   
        valor = Poda_AlfaBeta(*it, jugador, profundidad + 1, profundidad_max, c_piece, id_piece, dice, alpha, beta, heuristic);
        if (Max_verstappen) { 
            if (alpha < valor) { 
                alpha = valor;
                if (profundidad == 0) { 
                    c_piece = it.getMovedColor();
                    id_piece = it.getMovedPieceId();
                    dice = it.getMovedDiceValue();                  
                }
            }
            if (alpha >= beta) return beta;
        } else { 
            if (beta > valor) beta = valor;
            if (beta <= alpha) return alpha;
        }
    }
    return Max_verstappen ? alpha : beta;
}
\end{lstlisting}


\section{Heurística}

La heurística utilizada tiene las siguientes ponderaciones:

\begin{lstlisting}[language=C++]
double AIPlayer::MiValoracion1(const Parchis &estado, int jugador) {
    int ganador = estado.getWinner();
    int oponente = (jugador + 1) % 2;
    const int CASILLASRECORRER = 68 + 7; 

    if (ganador == jugador) return gana; 
    if (ganador == oponente) return pierde; 

    vector<color> my_colors = estado.getPlayerColors(jugador);
    vector<color> op_colors = estado.getPlayerColors(oponente);

    double puntuacion_jugador = 0.0;
    color max_goal_color;
    int max_pieces_at_goal = -1;

    for (int i = 0; i < my_colors.size(); i++) {
        color c = my_colors[i];
        int pieces_at_goal = estado.piecesAtGoal(c);
        if (pieces_at_goal > max_pieces_at_goal) {
            max_pieces_at_goal = pieces_at_goal;
            max_goal_color = c;
        }
    }

    for (int i = 0; i < my_colors.size(); i++) {
        color c = my_colors[i];
        puntuacion_jugador -= estado.piecesAtHome(c) * 5;
        int pieces_at_goal = estado.piecesAtGoal(c);
        puntuacion_jugador += pieces_at_goal * 100;
        if (pieces_at_goal == 2) puntuacion_jugador += 200; 

        for (int j = 0; j < num_pieces; j++) {
            int distance = estado.distanceToGoal(c, j);
            if (distance > 0) {
                double factor = (c == max_goal_color) ? 0.2 : 0.1; 
                puntuacion_jugador += (CASILLASRECORRER - distance) * factor;
            }
        }
    }

    if (estado.getCurrentPlayerId() == jugador) {
        if (estado.isEatingMove()) {
            pair<color, int> Comidas = estado.eatenPiece();
            puntuacion_jugador += (Comidas.first == my_colors[0] || Comidas.first == my_colors[1]) ? 10 : 50;
        }
        if (estado.isGoalMove()) puntuacion_jugador += 20;
        if (estado.goalBounce()) puntuacion_jugador -= 10; 
    }

    double puntuacion_oponente = 0.0;
    color max_goal_color_op;
    int max_pieces_at_goal_op = -1;

    for (int i = 0; i < op_colors.size(); i++) {
        color c = op_colors[i];
        int pieces_at_goal = estado.piecesAtGoal(c);
        if (pieces_at_goal > max_pieces_at_goal_op) {
            max_pieces_at_goal_op = pieces_at_goal;
            max_goal_color_op = c;
        }
    }

    for (int i = 0; i < op_colors.size(); i++) {
        color c = op_colors[i];
        puntuacion_oponente -= estado.piecesAtHome(c) * 5;
        int pieces_at_goal = estado.piecesAtGoal(c);
        puntuacion_oponente += pieces_at_goal * 100;
        if (pieces_at_goal == 2) puntuacion_oponente += 200;

        for (int j = 0; j < num_pieces; j++) {
            int distance = estado.distanceToGoal(c, j);
            if (distance > 0) {
                double factor = (c == max_goal_color_op) ? 0.2 : 0.1; 
                puntuacion_oponente += (CASILLASRECORRER - distance) * factor;
            }
        }
    }

    if (estado.getCurrentPlayerId() == oponente) {
        if (estado.isEatingMove()) {
            pair<color, int> Comidas = estado.eatenPiece();
            puntuacion_oponente += (Comidas.first == op_colors[0] || Comidas.first == op_colors[1]) ? 10 : 50;
        }
        if (estado.isGoalMove()) puntuacion_oponente += 20;
        if (estado.goalBounce()) puntuacion_oponente -= 10;
    }
    return puntuacion_jugador - puntuacion_oponente;
}
\end{lstlisting}

\section{Valoración de las piezas}

Para el jugador:
\begin{itemize}
    \item \textbf{Piezas en Casa}: Penalización de 5 puntos por cada pieza.
    \item \textbf{Piezas en la Meta}: 100 puntos por pieza, con un bono de 200 puntos si hay dos.
    \item \textbf{Distancia al Objetivo}: Bonificación basada en la cercanía al objetivo, con factores de ponderación de 0.2 y 0.1.
    \item \textbf{Movimientos Especiales}: Bonos y penalizaciones por capturas, movimientos a la meta y rebotes.
\end{itemize}

Para el oponente: 
Las valoraciones son idénticas.

\section{Ventajas y desventajas}

Las ventajas de la heurística son su equilibrio y la capacidad de hacer buenas jugadas, venciendo a los ninjas 0, 1 y 2. Sin embargo, no logra vencer al ninja 3 y a veces falla en capturar o evitar ser capturado.

\section{Menciones a las otras heurísticas}

Las otras heurísticas son evoluciones de las primeras. La segunda intentó un enfoque diferente y la tercera es una remodelación de la primera para intentar vencer al ninja 3.

\end{document}
